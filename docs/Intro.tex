Dynamical systems are a cornerstone of mathematical and physical sciences, providing profound insights into how systems evolve over time. Whether studying planetary orbits, stock market fluctuations, or the spread of a virus through a population, dynamical systems offer a robust framework for understanding complex processes. This mathematical concept not only enhances our understanding of the world but also equips us with tools to predict the future behavior of systems under specific conditions. The beauty of dynamical systems lies in their universal applicability, encompassing both deterministic systems—where the future behavior is entirely determined by initial conditions—and stochastic systems, where randomness plays a significant role in the system's evolution.

To study the evolution of such systems, various approaches, known as system identification methods, can be employed. These methods analyze a set of observed or latent states and aim to devise mathematical models to predict the system’s future states. Specifically, for identifying nonlinear dynamics, several deterministic and probabilistic tools are available, including radial basis functions \cite{Chen1990Non-linear}, neural networks \cite{cochocki1993neural}, Gaussian processes \cite{kocijan2005dynamic} \cite{raissi2017hidden} \cite{raissi2017inferring} \cite{raissi2017machine} \cite{raissi2017numerical}, and nonlinear auto-regressive models such as NARMAX \cite{billings2013nonlinear} and recurrent neural networks \cite{goodfellow2016deep}. However, achieving sparse representations of dynamics with these methods often requires the nontrivial task of selecting an appropriate set of basis functions. Thus, expanding the search space for functions is a key area of ongoing research.

In this project, we introduce a novel approach to nonlinear system identification, combining classical multistep time-stepping schemes from numerical analysis with deep neural networks. Inspired by recent advancements in physics-informed neural networks (PINNs), we propose a structured nonlinear regression model capable of uncovering dynamic dependencies from a given set of temporal data snapshots, ultimately returning a closed-form basin of attraction. Unlike recent approaches to system identification, our method does not require direct access to or approximations of temporal gradients, as time derivatives are discretized using traditional time-stepping methods. Additionally, we employ a broader family of function approximators, which eliminates the need to commit to a specific class of basis functions, such as polynomials.

This project is structured as follow. We start with the introduction of the lorenz systems and his chaotic behaviour Chp. \ref{cap: problem}.
In the next capters we will introduced numerical methods used for this problem Chp. \ref{cap: overview}, the problem setup with PINNs Chp. \ref{cap: setup} and code implementation \ref{cap: code}. 
